\documentclass[12pt]{article}
% Python Setup https://tex.stackexchange.com/questions/83882/how-to-highlight-python-syntax-in-latex-listings-lstinputlistings-command
\usepackage[utf8]{inputenc}
\usepackage{url}
\usepackage{tcolorbox}
\usepackage{biblatex}
% Default fixed font does not support bold face
\DeclareFixedFont{\ttb}{T1}{txtt}{bx}{n}{12} % for bold
\DeclareFixedFont{\ttm}{T1}{txtt}{m}{n}{12}  % for normal

% Custom colors
\usepackage{color}
\definecolor{deepblue}{rgb}{0,0,0.5}
\definecolor{deepred}{rgb}{0.6,0,0}
\definecolor{deepgreen}{rgb}{0,0.5,0}

\usepackage{listings}
\usepackage{biblatex}
\addbibresource{chapter5bib.bib}

% Python style for highlighting
\newcommand\pythonstyle{
\lstset{
language=Python,
basicstyle=\ttm,
otherkeywords={self},             % Add keywords here
keywordstyle=\ttb\color{deepblue},
emph={MyClass,__init__},          % Custom highlighting
emphstyle=\ttb\color{deepred},    % Custom highlighting style
stringstyle=\color{deepgreen},
frame=tb,                         % Any extra options here
showstringspaces=false            % 
}}


% Python environment
\lstnewenvironment{python}[1][breaklines]
{
\pythonstyle
\lstset{#1}
}
{}

% Python for external files
\newcommand\pythonexternal[2][]{{
\pythonstyle
\lstinputlisting[#1]{#2}}}

\bibliography{chapter5bib.bib}
% Python for inline
\newcommand\pythoninline[1]{{\pythonstyle\lstinline!#1!}}
\title{Python Tutorial on Class}
\author{Leo Zipeng Lin}
\begin{document}
\maketitle
\newpage
\begin{tcolorbox}[colback=blue!5!white,colframe=green!50!blue,title=Friendly Alert]
	Now we talk about python class.
\end{tcolorbox}
\section{Why do we need class?}

\begin{tcolorbox}[colback=blue!5!white,colframe=green!50!blue,title=Friendly Alert]
	\par If you did AP Computer Science A, you can probably skip this part (If you remember :) )
\end{tcolorbox}
\par Sometimes we need to operate through a same type of \textit{objects}. \cite{wiki:xxx}
For instance, if the school wants to 'take care" of the students', then it has to know the \textbf{Name}, \textbf{Age}, \textbf{Grade}, and other characters.
Of course, we are able to assign those 'character' one by one, like 
\pythonexternal{./Chapter5_code/One_by_one.py}

\par Apparently, this is \textbf{efficient} if there is 200 students.
We can think of a way like this: what if we just treat the people as \textbf{Objects}?
For instance, we treat all the students as 
"human beings with age number, gender, name, and other information", 
and we just create one 'class' for all of the students.
That's basically why class is so useful
\begin{center}
\textbf{To say this in another way, you can create your own type!}
\end{center}
\newpage
\section{Example}
\pythonexternal{./Chapter5_code/Class_Example.py}
\par Right now, we are able to assign each student with their characteristics. Consider the effect of creating a class when we need to write a system to sort students in their grade. Since the grade is assigned, we are able to access to the data easily. \cite{Python}
\section{Explanation for the code}
\begin{tcolorbox}[colback=blue!5!white,colframe=blue!50!white,title=Reader's Complain]
	Hey, we don't even understand the code!
\end{tcolorbox}
\par Don't worry, now we are going to go through the code.
\begin{python}
	class student(object):
\end{python}
\par In this line, we create a new \textbf{Class} called student.
\par Pay attention to \textbf{object}, now we are going to explain what this is.
%TODO:explain object
\begin{python}
	def __init__ (self,age,name,grade,gender):
\end{python}
<<<<<<< HEAD
\par In this line, we create a \textbf{function} (read the chapter before) called \textbf{init}, 
as known as the \textbf{initiate} function.
The init function is always executed when the class is being initiated.
For its usage, we always to this function to \textbf{assign values to object properties}. Here the 'init' function assign value for the student class.
\begin{python}
	self.name =name
	self.age=age
	self.grade=grade
	self.gender=gender
\end{python}
Here the \framebox{self} method is \textbf{a method to refer to the current instance, and \textit{access} variables that belong to the class}.
To be specific, the self parameter here accesses to the \textbf{name, age, grade, gender} variables. Here we set them to the name we give them in the \textbf{init} function. If you want to build a 'class' of student (pun intended) in tenth grade, you can set this instead:
\begin{python}
	self.grade=10
\end{python}
Isn't this easy!
\begin{tcolorbox}[colback=red!5!white,colframe=white!50!red,title=Reader's Complain]
	Ask: What about I try
	\begin{python}
		self.self=...
	\end{python}
	Answer: Of course you can try! However, only some cases work, can you figure out why it would work? \footnote[1]{the parameter after self.self has to be defined. For instance self.self=10 works, the function just calls itself.}
\end{tcolorbox}

\begin{python}
	def announcement (self):
\end{python}
\par This is a function, and the parameter is self, the student I created in the class function.
As it seems, the function wants to write an simple assignment system.
It basically writes some 'if' statements to figure out which grade the student is in. We are not going to talk about each like of code specifically, since it just uses the knowledge of 'if', 'elif' and 'else'.
\begin{python}
	leolin=student(12,"Leo Lin", 11, "Preferred not to tell")
\end{python}
\par This is a function to create a student with variable 'leolin'.
From this line, we could learn how to create a object from your own class.
\begin{python}
	leolin.announcement()
\end{python}
This is just a function, which calls for the announcement function created above.
% Future plan
\section{Inheritance}
Create a parent class, child class
% https://www.w3schools.com/python/python_inheritance.asp May be I will add it as a source
\par This is basically creating two classes, while one of them is from the other one class.
For example, we first create a 'Parent' Class:
\begin{python}
	class Student:
		def \_\_init\_\_(self, first-name,last-name, grade):
			self.firstname=first-name
			self.lastname=last-name
		def printname(self):
			print(self.firstname, self.lastname)
\end{python}
\par Now we create the children class called 'junior'
\begin{python}
	class Junior(Student):
		pass
\end{python}
\par Here the 'pass' keyword is use because I don't want to add any other properties to the Junior Class.
\par Like the \texttt{Student} class, the \texttt{Junior} class could also use the function print name
\par We can also use the \_\_init\_\_ Function here, while there would be another function called \texttt{super}
\begin{python}
	class Junior (Student):
		def \_\_init\_\_(self, first-name, last-name, grade, year):
		super().\_\_init\_\_(fist-name, last-name, grade)
		self.graduationyear=year
\end{python}
\par Here the properties from the parents are inherited.
\subsection{Add properties}
\subsection{Add methods}
\section{Iterators}
\section{Scope}
\section{Modules}

%TODO: explain
\par In this line, we create a \textbf{function} (read the chapter before) called \textbf{init}. It is used to \textbf{initiate} the object. All the classes have a function called 'init', which is always executed when the class is being initiated. When we execute this function, we do the operations to assign \textbf{basic values}. Compared to other functions, this one is executed \textit{automatically}.
%TODO: explain
\begin{python}
  self.name=name	
	self.age=age
	self.grade=grade
	self.gender=gender
\end{python}
\par Here, the \textbf{self} is a parameter that refers to the current object of the class, here it is the \textbf{student}. It does not have to be the word called 'self', we can also use \textbf{any word you want}.
\section{Exercises}
\begin{enumerate}
	\item Try to write a \textbf{object} for a country
		\begin{itemize}
			\item Try to write a function in case to create a country (name, population)
			\item Write a function to update the population (like after the cencus)
		\end{itemize}
\end{enumerate}
\newpage
\section{Answers/Hints to the exercises}
The answers are just for refrences.
\begin{python}
class country:
	def \_\_init\_\_(self, name, population):
		self.name=name
		self.population=population
	def update(self, new-population):
		self.popultion=new-population
	# Check whether it works
jp=country("Japan", 10000000)
jp.update(2000000)
\end{python}
\newpage
\printbibliography
\end{document}

