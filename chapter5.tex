\documentclass[12pt]{article}
% Python Setup https://tex.stackexchange.com/questions/83882/how-to-highlight-python-syntax-in-latex-listings-lstinputlistings-command
\usepackage[utf8]{inputenc}
\usepackage{url}
\usepackage{tcolorbox}
% Default fixed font does not support bold face
\DeclareFixedFont{\ttb}{T1}{txtt}{bx}{n}{12} % for bold
\DeclareFixedFont{\ttm}{T1}{txtt}{m}{n}{12}  % for normal

% Custom colors
\usepackage{color}
\definecolor{deepblue}{rgb}{0,0,0.5}
\definecolor{deepred}{rgb}{0.6,0,0}
\definecolor{deepgreen}{rgb}{0,0.5,0}

\usepackage{listings}

% Python style for highlighting
\newcommand\pythonstyle{\lstset{
language=Python,
basicstyle=\ttm,
otherkeywords={self},             % Add keywords here
keywordstyle=\ttb\color{deepblue},
emph={MyClass,__init__},          % Custom highlighting
emphstyle=\ttb\color{deepred},    % Custom highlighting style
stringstyle=\color{deepgreen},
frame=tb,                         % Any extra options here
showstringspaces=false            % 
}}


% Python environment
\lstnewenvironment{python}[1][]
{
\pythonstyle
\lstset{#1}
}
{}

% Python for external files
\newcommand\pythonexternal[2][]{{
\pythonstyle
\lstinputlisting[#1]{#2}}}

% Python for inline
\newcommand\pythoninline[1]{{\pythonstyle\lstinline!#1!}}
\title{Python Tutorial on Class}
\author{Leo Zipeng Lin}
\begin{document}
\maketitle
\newpage
\begin{tcolorbox}[colback=blue!5!white,colframe=green!50!blue,title=Friendly Alert]
	Now we talk about python class.
\end{tcolorbox}
\section{Why do we need class?}

\begin{tcolorbox}[colback=blue!5!white,colframe=green!50!blue,title=Friendly Alert]
	\par If you did AP Computer Science A, you can probably skip this part (If you remember :) )
\end{tcolorbox}
\par Sometimes we need to operate through a same type of \textit{objects}. 
For instance, if the school wants to 'take care" of the students', then it has to know the \textbf{Name}, \textbf{Age}, \textbf{Grade}, and other characters.
Of course, we are able to assign those 'character' one by one, like 
\pythonexternal{./Chapter5_code/One_by_one.py}

\par Apparently, this is \textbf{efficient} if there is 200 students.
We can think of a way like this: what if we just treat the people as \textbf{Objects}?
For instance, we treat all the students as 
"human beings with age number, gender, name, and other information", 
and we just create one 'class' for all of the students.
That's basically why class is so useful
\begin{center}
\textbf{To say this in another way, you can create your own type!}
\end{center}
\newpage
\section{Example}
\pythonexternal{./Chapter5_code/Class_Example.py}
\par Right now, we are able to assign each student with their characteristics. Consider the effect of creating a class when we need to write a system to sort students in their grade. Since the grade is assigned, we are able to access to the data easily.
\section{Explanation for the code}
\begin{tcolorbox}[colback=blue!5!white,colframe=blue!50!white,title=Reader's Complain]
	Hey, we don't even understand the code!
\end{tcolorbox}
\par Don't worry, now we are going to go through the code.
\begin{python}
	class student(object):
\end{python}
\par In this line, we create a new \textbf{Class} called student.
\par Pay attention to \textbf{object}, now we are going to explain what this is.
%TODO:explain object
\begin{python}
	def __init__ (self , age , name , grade , gender ):
\end{python}
\par In this line, we create a \textbf{function} (read the chapter before) called \textbf{__init__}.
%TODO: explain
\section{Further Reference}
\bibliographystyle{plain}
\bibliography{chapter5bib.bib}
\cite{Python}
\end{document}
